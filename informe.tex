\documentclass[titlepage,a4paper]{article}

\usepackage{a4wide}
\usepackage[colorlinks=true,linkcolor=black,urlcolor=blue,bookmarksopen=true]{hyperref}
\usepackage{bookmark}
\usepackage{fancyhdr}
\usepackage[spanish]{babel}
\usepackage[utf8]{inputenc}
\usepackage[T1]{fontenc}
\usepackage{graphicx}
\usepackage{float}

\pagestyle{fancy} % Encabezado y pie de página
\fancyhf{}
\fancyhead[L]{TP1}
\fancyhead[R]{FIUBA}
\renewcommand{\headrulewidth}{0.4pt}
\fancyfoot[C]{\thepage}
\renewcommand{\footrulewidth}{0.4pt}

\begin{document}
\begin{titlepage} % Carátula
    \centering
    \vfill
    \Huge \textbf{Trabajo Práctico 1}
    \vskip2cm
    \Large [66.20] Organizacion de la Computadora\\
    Primer cuatrimestre de 2019 
    \vskip2cm
    \begin{table}[htbp]
	\begin{center}
	\begin{tabular}{|l|l|l|}
	\hline
    \multicolumn{3}{|c|}{Grupo 2} \\ \hline
	Nombre & Padrón & Mail \\ \hline 
    Tomas Lopez Hidalgo & & \\ \hline 
    Eliana Gamarra & 100016 & elianagam2@gmail.com\\ \hline
    Leonardo Bellaera &  100973 & leobellaera@gmail.com \\ \hline
    \end{tabular}
	\label{tabla:sencilla}
	\end{center}
	\end{table}

    \vfill
\end{titlepage}
\tableofcontents % Índice general
\newpage

\section{Introduccion}\label{sec:intro}


\section{Desarrollo}\label{sec:intro}
Con la intención de familiarizarnos con el emulador GXemul -un emulador de la arquitectura de computadores MIPS- y NetBSD, desarrollamos dos programas en código C cuyas funciones son parsear archivos de texto UNIX a Windows y viceversa. Llamamos unix2dos al programa encargado de transformar los archivos de texto de UNIX a Windows, y dos2unix al encargado de realizar la operación inversa.
Ambos fueron compilados con el compilador gcc (...) %terminar
También se obtuvieron sus respectivos códigos Assembler en MIPS (que luego serán traducidos en código de máquina para ser interpretados por el emulador de la arquitectura MIPS)

\subsection{Preparando el emulador}

Se ejecutaron los siguientes comandos para poder correr NetBSD bajo el emulador:

sudo su 
useradd -m gxemul
passwd gxemul


\end{document}