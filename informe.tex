\documentclass[titlepage,a4paper]{article}

\usepackage{a4wide}
\usepackage[colorlinks=true,linkcolor=black,urlcolor=blue,bookmarksopen=true]{hyperref}
\usepackage{bookmark}
\usepackage{fancyhdr}
\usepackage[spanish]{babel}
\usepackage[utf8]{inputenc}
\usepackage[T1]{fontenc}
\usepackage{graphicx}
\usepackage{float}

\pagestyle{fancy} % Encabezado y pie de página
\fancyhf{}
\fancyhead[L]{TP1}
\fancyhead[R]{FIUBA}
\renewcommand{\headrulewidth}{0.4pt}
\fancyfoot[C]{\thepage}
\renewcommand{\footrulewidth}{0.4pt}

\begin{document}
\begin{titlepage} % Carátula
    \centering
    \vfill
    \Huge \textbf{Trabajo Práctico 1}
    \vskip2cm
    \Large [66.20] Organizacion de la Computadora\\
    Primer cuatrimestre de 2019 
    \vskip2cm
    \begin{table}[htbp]
	\begin{center}
	\begin{tabular}{|l|l|l|}
	\hline
    \multicolumn{3}{|c|}{Grupo 2} \\ \hline
	Nombre & Padrón & Mail \\ \hline 
    Tomas Lopez Hidalgo & & \\ \hline 
    Eliana Gamarra & 100016 & elianagam2@gmail.com\\ \hline
    Leonardo Bellaera &  100973 & leobellaera@gmail.com \\ \hline
    \end{tabular}
	\label{tabla:sencilla}
	\end{center}
	\end{table}

    \vfill
\end{titlepage}
\tableofcontents % Índice general
\newpage

\section{Introduccion}\label{sec:intro}


\section{Desarrollo}\label{sec:intro}
Con la intención de familiarizarnos con el emulador GXemul -un emulador de la arquitectura de computadores MIPS- y NetBSD, desarrollamos dos programas en código C cuyas funciones son parsear archivos de texto UNIX a Windows y viceversa. Llamamos unix2dos al programa encargado de transformar los archivos de texto de UNIX a Windows, y dos2unix al encargado de realizar la operación inversa.
Ambos fueron compilados con el compilador gcc (...) %terminar
También se obtuvieron sus respectivos códigos Assembler en MIPS (que luego serán traducidos en código de máquina para ser interpretados por el emulador de la arquitectura MIPS)

\subsection{Configuracion del entorno de trabajo}

Con la intención de familiarizarnos con el emulador Gxemul de un procesador MIPS se crearon 2 archivos en c:
dos2unix
unix2dos
Ambos con la extensión .c se compilaron con gcc y generando un ejecutable con el mismo nombre de archivo
También dió como resultado sus respectivos códigos assembler en MIPS que luego se traducirá a código de máquina (binario)

Se subieron los siguientes pasos para poder ejecutar NetBSD

sudo su
useradd -m gxemul
passwd gxemul

Una vez configurado el usuario, se agrega una dirección IP con el comando:
ifconfig lo:0 172.20.0.1, para la máquina de emulacion de MIPS que se corren con la imagen proporcionada por la catedra que contiene la imagen de NetBSD
\begin{lstlisting}
$ ./gxemul -e 3max -d netbsd-pmax.img
 completando user y pass con
	user : root
	contraseña : orga6620

\end{lstlisting}
Para conectar nuestra consola de NETBSD  a Linux se hace un puente con ssh, dentro de la imagen de NetBSD abierta con el anterior comando
$ ssh -R 2222:127.0.0.1:22 <userLinux>@172.20.0.1	 (user debe reemplazarse por el usuario de la sesión actual)

Desde una nueva consola de Linux (en la que no está abierto NetBSD) $ ssh -p 2222 root@127.0.0.1 en donde nos va a pedir la contraseña de root (que es orga6620). Una vez aceptado y elegimos la terminal xterm
	
Entonces hasta ahora lo que creamos es una ventana (por la cual miramos) desde NetBSD (1° consola) a Linux (vemos como si estuviesemos en Linux); y otra de Linux a NetBSD (2° consola) vemos el usuario root.

Para poder copiar archivos que .c que programamos en linux vamos a hacer 
\begin{lstlisting}
scp -p2222 [-r: para carpetas] origen destino

\end{lstlisting}
Voy a la consola en que "veo" Linux (1° consola, aquella en que abri NetBSD)

\begin{lstlisting}
$ "scp -p2222 ARCHIVO root@127.0.0.1:/DIRECTORIO/"
\end{lstlisting}
(notar que después de root@ lo que aparece es la ip que definimos antes en Linux)


la contraseña que deberé poner es la del OTRO sistema el cuál aparece en bash, si lo hago por la forma 1 entonces uso la pass de root@, en la 2da uso la de Linux.


\end{document}
